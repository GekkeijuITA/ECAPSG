\documentclass[12pt, a4paper]{report}
\usepackage[italian]{babel}

\begin{document}
\begin{titlepage}
    \title{Soluzioni Quiz}
    \author{}
    \date{}
\end{titlepage}
\maketitle
\tableofcontents
\chapter{Unità 1}
    \textit{Le domande cambiano ordine ad ogni tentativo.}
    \begin{itemize}
        \item \textbf{La cd. Legge Fornero n. 92/2912}: fu emanata su impulso dei mercati finanziari.
        \item \textbf{Con il termine flexicurity si intende}: l’insieme delle politiche del lavoro finalizzate ad assicurare la mobilità all’interno del mercato del lavoro e a sostenere il reddito dei lavoratori nei periodi di transizione da un lavoro all’altro.
        \item \textbf{Il contratto collettivo di diritto comune}: acquista efficacia erga omnes nella parte in cui la fonte giurisprudenziale estende i trattamenti minimi alla totalità dei rapporti di lavoro.
        \item \textbf{La cd. Legge La Pergola}: prevede l’obbligo per il Parlamento italiano di recepire annualmente i contenuti delle direttive europee.
        \item \textbf{Il contratto collettivo come fonte del diritto del lavoro}: è sullo stesso livello della legge.
        \item \textbf{Il sistema delle fonti del diritto del lavoro}: è costituito da un complesso di norme derivanti sia dall’autonomia dei privati che dall’ordinamento legislativo.
    \end{itemize}
\chapter{Unità 2}
    \textit{Le domande cambiano ordine ad ogni tentativo.}
    \begin{itemize}
        \item \textbf{Con lo Statuto dei lavoratori}: Si rafforzarono le garanzie e le tutele dei lavoratori e delle oo.ss.
        \item \textbf{Il contratto di lavoro a tutele crescenti}: introduce la libera recedibilità nei primi tre anni, tranne che per i licenziamenti discriminatori.
        \item \textbf{Il rapporto di lavoro}: ha natura contrattuale.
        \item \textbf{Rispetto al contratto a progetto, il Jobs Act}: propone il superamento delle collaborazioni a progetto.
        \item \textbf{Il contratto di lavoro a tutele crescenti}: È un contratto di lavoro a tempo indeterminato.
        \item \textbf{La legge Fornero n. 92/2012}: Limitò le co.co.pro. nel lavoro privato
    \end{itemize}
\chapter{Unità 3}
    \textit{Le domande cambiano ordine ad ogni tentativo.}
    \begin{itemize}
        \item \textbf{Il contratto collettivo nel pubblico impiego}: ammesso solo per le materie contrattualizzate.
        \item \textbf{La pubblica amministrazione nei confronti dei dipendenti}: opera alla stregua del datore di lavoro privato.
        \item \textbf{I contratti collettivi di comparto producono effetti}: erga omnes, a seguito della pubblicazione in Gazzetta Ufficiale.
        \item \textbf{Le pubbliche amministrazioni possono assumere}: per pubblico concorso.
        \item \textbf{I vincitori di un concorso pubblico}: hanno un diritto soggettivo pieno all’assunzione.
        \item \textbf{L’ARAN è}: un’autorità amministrativa indipendente.
        \item \textbf{Le categorie di dipendenti escluse dalla cd. contrattualizzazione sono}: coloro che rappresentano le funzioni vitali dello Stato.
    \end{itemize}
\chapter{Unità 4}
    \textit{Le domande cambiano ordine ad ogni tentativo.}
    \begin{itemize}
        \item \textbf{La nuova disciplina del contratto a termine introdotta con il cd. Decreto Poletti si propone di}: Rendere più facile il ricorso al contratto di lavoro a tempo determinato per soddisfare le esigenze temporanee delle imprese, arginando gli abusi attraverso l’aumento degli oneri previdenziali e l’obbligo di assicurare un rapporto tra lavoratori a termine e quelli a tempo indeterminato che non ecceda il 20 per cento.
        \item \textbf{La somministrazione di lavoro a tempo indeterminato (staff Leasing)}: è ammessa sempre.
        \item \textbf{Il ricorso al contratto di lavoro a termine nel settore pubblico si differenzia da quello privato in quanto}: Nella pubblica amministrazione il contratto a termine può essere stipulato solo per soddisfare esigenze eccezionali o temporanee, che siano vagliate per il tramite di concorso pubblico.
        \item \textbf{Le clausole di contingentamento}: sono affidate alla legge.
        \item \textbf{Se due assunzioni a termine avvengono senza interruzione, il contratto}: si considera a tempo indeterminato ex tunc.
        \item \textbf{L’abuso di lavoro flessibile nella p.a. è sanzionato}: con il risarcimento del danno.
        \item \textbf{La somministrazione a tempo determinato}: è ammessa entro gli stessi limiti di durata del contratto a termine.
    \end{itemize}
\chapter{Unità 5}
    \textit{Le domande cambiano ordine ad ogni tentativo.}
    \begin{itemize}
        \item \textbf{IL contratto di lavoro full-time può essere trasformato in part-time}: in virtù di un accordo tra le parti.
        \item \textbf{Il diritto alle ferie, garantito dall’art. 36, co. 3 Cost è}: Un diritto irrinunciabile, la cui monetizzazione può avvenire solo alla cessazione del rapporto e qualora il lavoratore ha maturato ferie non godute.
        \item \textbf{Quali sono i limiti posti dalla legge alla durata della prestazione di lavoro?}: limiti all’orario giornaliero di lavoro sono quelli derivanti dall’obbligo di rispettare la durata massima di 48 ore di lavoro settimanali e il diritto al riposo giornaliero di 11 ore, oltre alla pausa di 10 minuti se la prestazione giornaliera supera le sei ore.
        \item \textbf{Il lavoro festivo}: richiede il pagamento di una maggiorazione retributiva e un riposo compensativo settimanale.
        \item \textbf{Il contratto di lavoro a tempo parziale è una tipologia contrattuale}: La cui prestazione lavorativa è ridotta nel tempo e modulata in base ai giorni, ai mesi o all’anno. L’orario di lavoro del part-timer deve essere concordato per iscritto e può essere modificato, a mezzo di clausole elastiche o flessibili, nel corso del rapporto di lavoro per ragioni di carattere aziendale previo consenso del lavoratore.
        \item \textbf{L’orario di lavoro straordinario è determinato}: dai contratti collettivi di lavoro.
    \end{itemize}
\chapter{Unità 6}
    \textit{Le domande cambiano ordine ad ogni tentativo.}
    \begin{itemize}
        \item \textbf{I congedi parentali}: Consentono ad entrambi i genitori di astenersi dal lavoro per un periodo massimo di dieci mesi complessivi con diritto ad un’indennità integrativa pari al 30\% della retribuzione.
        \item \textbf{Le clausole di nubilato}: vietavano alla donna di contrarre matrimonio durante il rapporto di lavoro.
        \item \textbf{La legge delega sul Jobs Act in materia di tutela della genitorialità}: Incentiva la contrattazione collettiva integrativa a introdurre meccanismi di flessibilità nella gestione dei congedi e dell’orario di lavoro.
        \item \textbf{Le lavoratrici a progetto}: godono del diritto al congedo di maternità indipendentemente dal versamento dei contributi previdenziali.
        \item \textbf{Il congedo di maternità è}: un periodo di astensione obbligatoria dal lavoro da parte della lavoratrice madre.
    \end{itemize}
\chapter{Unità 7}
    \textit{Le domande cambiano ordine ad ogni tentativo.}
    \begin{itemize}
        \item \textbf{L’assegnazione a qualifiche e/o mansioni superiori nel pubblico impiego}: è consentita in presenza di determinate condizioni.
        \item \textbf{Secondo la disciplina dell’art. 2103 c.c., come sostituito dall’art. 3 del d.lgs. 11 gennaio 2015, in caso di assegnazione a mansioni superiori il lavoratore}: ha diritto alla retribuzione corrispondente alle mansioni svolte e acquista il diritto all’assegnazione definitiva delle superiori mansioni dopo un periodo di adibizione stabilito dai contratti collettivi, o in mancanza di questi ultimi non superiore a sei mesi, salvo il caso in cui l’assegnazione sia stata disposta per sostituire un lavoratore con diritto alla conservazione del posto di lavoro.
        \item \textbf{Il mutamento di mansioni richiede}: il rispetto della categoria legale.
        \item \textbf{I quadri sono}: lavoratori subordinati che, pur non appartenendo alla categoria dei dirigenti, svolgono funzioni a carattere continuativo di rilevante importanza ai fini dello sviluppo e dell’attuazione degli obiettivi dell’impresa.
        \item \textbf{La disciplina del nuovo d. lgs. n. 81/2015}: consente l’adibizione a qualifiche inferiori quando lo prevede il contratto collettivo?
        \item \textbf{La qualifica professionale}: determina il livello della retribuzione.
    \end{itemize}
\chapter{Unità 8}
    \textit{Le domande cambiano ordine ad ogni tentativo.}
    \begin{itemize}
        \item \textbf{Il trattamento di fine rapporto viene corrisposto}: In ogni caso di cessazione del rapporto o anche in via anticipata, nella misura del 70 per cento del trattamento maturato fino a quel momento, quando il lavoratore ne faccia richiesta per gravi ragioni.
        \item \textbf{La retribuzione a cottimo è}: una modalità retributiva di natura premiale.
        \item \textbf{Cosa è il cuneo fiscale}: l’insieme delle voci della retribuzione destinate alla tassazione e alla contribuzione sociale.
        \item \textbf{L’obbligo della retribuzione è disciplinato dalla legge la quale dispone}: Che la retribuzione deve essere proporzionale e sufficiente all’attività di lavoro svolta, rinviando alla contrattazione collettiva la determinazione dell’entità dei trattamenti retributivi.
        \item \textbf{Il trattamento di fine rapporto può essere anticipato}: nella misura del 70 per cento del trattamento maturato fino a quel momento e solo per cause tassativamente elencate dalla legge.
        \item \textbf{I premi costituiscono quella parte della retribuzione}: A carattere variabile, dipendente dalla produttività del lavoro ovvero dalla redditività dell’azienda: nel primo caso l’attribuzione del premio è direttamente connessa all’impegno dei lavoratori, nel secondo caso segue l’andamento di fattori economici esterni all’impresa.
        \item \textbf{Il lavoro gratuito è}: ammesso in determinate circostanze.
    \end{itemize}
\chapter{Unità 9}
    \textit{Le domande cambiano ordine ad ogni tentativo.}
    \begin{itemize}
        \item \textbf{L’obbligo di dare il preavviso di licenziamento, ovvero di corrispondere la relativa indennità sostitutiva, non è dovuto}: in caso di licenziamento determinato da grave inadempimento degli obblighi contrattuali.
        \item \textbf{Il licenziamento disciplinare}: è un licenziamento per giusta causa.
        \item \textbf{Secondo la disciplina del c.d. contratto a tutele crescenti per i nuovi assunti, in caso di licenziamento per giustificato motivo oggettivo accertato dal giudice come inesistente il datore è tenuto}: esclusivamente al pagamento di una indennità commisurata all’anzianità di servizio.
        \item \textbf{Le dimissioni in bianco costituiscono}: dimissioni acquisite dal datore all’atto dell’assunzione prive di data.
        \item \textbf{A seguito della c.d. riforma Fornero dell’art. 18, l. n. 300/1970, in caso licenziamento intimato oralmente si applica}: a tutela reintegratoria forte indipendentemente dal numero dei dipendenti del datore.
        \item \textbf{La tutela indennitaria forte}: esclude il diritto alla reintegrazione.
        \item \textbf{Organizzazioni di tendenza}: vige il principio della libera recedibilità.
    \end{itemize}
\chapter{Unità 10}
    \textit{Le domande cambiano ordine ad ogni tentativo.}
    \begin{itemize}
        \item \textbf{Il procedimento ex art. 28 st. lav.}: è un procedimento monitorio.
        \item \textbf{I diritti sindacali erano riconosciuti dallo Statuto dei lavoratori e ora sono stati abrogati dal Jobs Act: tutti o solo in parte?}: i diritti sindacali sono tuttora disciplinati dallo Statuto dei lavoratori. Il Jobs Act è intervenuto su alcune disposizioni particolari dello Statuto.
        \item \textbf{Il datore di lavoro partecipa all’assemblea}: non ha alcun diritto a intervenire in assemblea.
        \item \textbf{Rappresentanza e rappresentatività sono la stessa cosa?}: no. Indica la differenza.
        \item \textbf{La differenza tra RSA e RSU concerne}: le modalità di elezione e la composizione interna.
        \item \textbf{Il procedimento ex art. 28 st. lav.}: è applicabile anche ai rapporti di lavoro con le pubbliche amministrazioni.
        \item \textbf{Il sindacato ha natura giuridica di associazione non riconosciuta?}: il più delle volte ma sono possibili anche altre forme giuridiche, nel rispetto comunque del principio della libertà di organizzazione sindacale (art. 39 Cost.).
        \item \textbf{Cosa si intende per condotta antisindacale “plurioffensiva”?}: condotta che offende una pluralità di beni tutelati.
    \end{itemize}
\chapter{Unità 11}
    \textit{Le domande cambiano ordine ad ogni tentativo.}
    \begin{itemize}
        \item \textbf{Il sistema di contrattazione collettiva è centralizzato, decentrato o di prossimità?}: è articolato sul doppio livello nazionale e aziendale/territoriale ma i contratti di livello decentrato (aziendale/territoriale) si applicano entro limiti e condizioni stabilite in ambito nazionale. Gli accordi di prossimità sono stati introdotti dall’art. 8, d.l. n. 138/2011 (conv. in l. n. 148/2011), possono essere stipulati a livello aziendale e/o territoriale in materie determinate e nel rispetto di precise finalità ma non costituiscono un livello autonomo di contrattazione.
        \item \textbf{Secondo l’art. 2113 c.c., le rinunzie e le transazioni dei diritti dei lavoratori sono}: annullabili.
        \item \textbf{Qual è la differenza tra contrattazione collettiva e concertazione?}: costituiscono ambedue modalità di estrinsecazione dell’attività sindacale ma concorrono nella concertazione anche soggetti privati e pubblici diversi dal sindacato, inoltre, la funzione è diversa.
        \item \textbf{La funzione derogatoria del contratto di prossimità}: si esprime indifferentemente in pejus o in melius.
        \item \textbf{Fra contratto collettivo di lavoro e contratto individuale si instaura un rapporto}: di integrazione, solo se in melius.
        \item \textbf{Il contratto collettivo ha efficacia erga omnes?}: L’art. 39 Cost., seconda parte, non ha trovato attuazione e il contratto collettivo è efficace nei confronti degli iscritti alle associazioni stipulanti, ovvero nei riguardi di coloro che, iscrivendosi all’organizzazione sindacale, hanno conferito ad essa il potere di rappresentanza per la stipulazione dei contratti collettivi.
    \end{itemize}
\chapter{Unità 12}
    \textit{Le domande cambiano ordine ad ogni tentativo.}
    \begin{itemize}
        \item 
    \end{itemize}
\end{document}