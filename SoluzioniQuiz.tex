\documentclass[12pt, a4paper]{report}
\usepackage[italian]{babel}

\begin{document}
\begin{titlepage}
    \title{Soluzioni Quiz}
    \author{}
    \date{}
\end{titlepage}
\maketitle
\tableofcontents
\chapter{Unità 1}
\textit{Le domande cambiano ordine ad ogni tentativo.}
\begin{itemize}
    \item \textbf{La cd. Legge Fornero n. 92/2912}: fu emanata su impulso dei mercati finanziari.
    \item \textbf{Con il termine flexicurity si intende}: l’insieme delle politiche del lavoro finalizzate ad assicurare la mobilità all’interno del mercato del lavoro e a sostenere il reddito dei lavoratori nei periodi di transizione da un lavoro all’altro.
    \item \textbf{Il contratto collettivo di diritto comune}: acquista efficacia erga omnes nella parte in cui la fonte giurisprudenziale estende i trattamenti minimi alla totalità dei rapporti di lavoro.
    \item \textbf{La cd. Legge La Pergola}: prevede l’obbligo per il Parlamento italiano di recepire annualmente i contenuti delle direttive europee.
    \item \textbf{Il contratto collettivo come fonte del diritto del lavoro}: è sullo stesso livello della legge.
    \item \textbf{Il sistema delle fonti del diritto del lavoro}: è costituito da un complesso di norme derivanti sia dall’autonomia dei privati che dall’ordinamento legislativo.
\end{itemize}
\end{document}